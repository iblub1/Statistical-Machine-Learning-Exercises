\newif\ifvimbug
\vimbugfalse

\ifvimbug
\begin{document}
\fi

\exercise{Linear Algebra Refresher}
 

\begin{questions}

%----------------------------------------------

\begin{question}{Matrix Properties}{5}
A colleague of yours suggests matrix addition and multiplication are similar to scalars, thus commutative, distributive and associative properties can be applied.
Prove if matrix addition and multiplication are commutative and associative analytically or give counterexamples. 
Is matrix multiplication distributive with respect to matrix addition? 
Again, prove it analytically or give a counterexample.
Considering three matrices $ A, B, C$ of size $n\times n$.

\begin{answer}
To answer the questions examples calculated by following matrices will be used:


\[
A=
\begin{bmatrix}
2 & -1  \\
0 &  1 
\end{bmatrix}\quad
B=
\begin{bmatrix}
1 & 0  \\
4 & 3 
\end{bmatrix}\quad
C=
\begin{bmatrix}
2 & 2  \\
2 & 2 
.
\end{bmatrix}
\]

	
The commutative property for matrix addition states: $A+B = B + A$.
\begin{equation}
	A + B = ( \begin{array}{c c} 
		3 & -1 \\
		4 & 4 \end{array} )
\end{equation}
\begin{equation}
B + A = ( \begin{array}{c c} 
3 & -1 \\
4 & 4 \end{array} ) = A + B
\end{equation}

The commutative property for matrix multiplacation states: $A \cdot B = B \cdot A$

\begin{equation}
	A \cdot B = \begin{bmatrix}
	-2 & -3 \\
	4 & 3\\
	\end{bmatrix}
\end{equation}

\begin{equation}
B \cdot A = \begin{bmatrix}
2 & -1 \\
8 & -1\\
\end{bmatrix}
\end{equation}

Thus $A \cdot B \neq B \cdot A$

The distributiv property for matrices states: $A\cdot B + A\cdot C = A (B+C)$ 


\[
\begin{bmatrix}
-2 & -3  \\
04 &  3 
\end{bmatrix}\quad
+
\begin{bmatrix}
2 & 2  \\
2 & 2 
\end{bmatrix}\quad
=
\begin{bmatrix}
2 & -1  \\
0 & 1 
\end{bmatrix}
\cdot
\begin{bmatrix}
3 & 2  \\
6 & 5
\end{bmatrix}
\]

\[
\begin{bmatrix}
0 & -1  \\
6 & 5 
\end{bmatrix}\quad
=
\begin{bmatrix}
0 & -1  \\
6 & 5
\end{bmatrix}
\]


\end{answer}

\end{question}

%----------------------------------------------

\begin{question}{Matrix Inversion}{6}
Given the following matrix 
\begin{equation*}
     A = ( \begin{array}{c c c} 
     1 & 2 & 3 \\
     1 & 2 & 4 \\
     1 & 4 & 5 \end{array} )
\end{equation*}
analytically compute its inverse $ A^{-1}$ and illustrate the steps.

If we change the matrix in
\begin{equation*}
     A = ( \begin{array}{c c c} 
     1 & 2 & 3 \\
     1 & 2 & 4 \\
     1 & 2 & 5 \end{array} )
\end{equation*}
is it still invertible? Why?

\begin{answer}\end{answer}

\end{question}
	
%----------------------------------------------

\begin{question}{Matrix Pseudoinverse}{3}
	Write the definition of the right and left Moore-Penrose pseudoinverse of a generic matrix $A \in \R^{n\times m}$.
	
	Given $A \in \R^{2 \times 3}$, which one does exist? Write down the equation for computing it, specifying the dimensionality of the matrices in the intermediate steps.
	
\begin{answer}\end{answer}
\end{question}

%----------------------------------------------

\begin{question}{Eigenvectors \& Eigenvalues}{6}
What are eigenvectors and eigenvalues of a matrix $A$? Briefly explain why they are important in Machine Learning.

\begin{answer}\end{answer}

\end{question}

%----------------------------------------------

\end{questions}
