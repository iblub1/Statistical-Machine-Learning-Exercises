\newif\ifvimbug
\vimbugfalse

\ifvimbug
\begin{document}
\fi

\exercise{Density Estimation}
In this exercise, you will use the datasets \texttt{densEst1.txt} 
and \texttt{densEst2.txt}. The datasets contain 2D data belonging
to two classes, $C_1$ and $C_2$.

\begin{questions}

%----------------------------------------------

\begin{question}{Gaussian Maximum Likelihood Estimation}{10}
Derive the ML estimate for the mean and covariance of the \textbf{multivariate} Gaussian distribution. Start your derivations with the function you optimize. Assume that you can collect i.i.d data. (Hint: you can find many matrix identities on the Matrix Cookbook (\url{https://www.math.uwaterloo.ca/~hwolkowi/matrixcookbook.pdf}) and at \url{http://en.wikipedia.org/wiki/Matrix_calculus}.)

\begin{answer}\end{answer}

\end{question}

%----------------------------------------------

\begin{question}{Prior Probabilities}{2}
Compute the prior probability of each class from the dataset. 

\begin{answer}
By counting the class frequencies in the given dataset we can obtain the prior probabilities.\\
\begin{align}
&p(C_{1})=\frac{239}{239+761}=0.239
&p(C_{2})=\frac{239}{239+761}=0.761
\end{align}
\end{answer}

\end{question}


%----------------------------------------------

\begin{question}{Biased ML Estimate}{5}
Define the bias of an estimator and write how we can compute it.
Then calculate the biased and unbiased estimates of the conditional distribution $p(x|C_i)$, assuming that each class can be modeled with a Gaussian distribution. Which parameters have to be calculated?
Show the final result and attach a snippet of your code.
Do not use existing functions, but rather implement the computations by yourself!

\begin{answer}
The bias of an estimator $\Theta$ measures the difference between the expected value of an estimator and its' true value. It is is defined as:
\begin{equation*}
Bias(\widehat{\theta})= E(\widehat{\theta}) - \theta
\end{equation*}
Assuming a Gaussian distribution, the biased and unbiased estimates of the conditional distributions can be calculated using the following formula:
\begin{align*}
&\overline{\mathbf{x}}_{unbiased}= \frac{1}{N} \sum_{i=1}^{n} \mathbf{x}_{i} \\
&\widehat{\boldsymbol{\Sigma}}_{unbiased}=\frac{1}{N-1} \sum_{i=1}^{N}(\mathbf{x}_{i}-\overline{\mathbf{x}})(\mathbf{x}_{i}-\overline{\mathbf{x}})^{\mathrm{T}} \\
&\widehat{\boldsymbol{\Sigma}}_{biased}=\frac{1}{N} \sum_{i=1}^{N}(\mathbf{x}_{i}-\overline{\mathbf{x}})(\mathbf{x}_{i}-\overline{\mathbf{x}})^{\mathrm{T}}
\end{align*}
We thus obtain the following values for the given dataset.
\begin{align*}
&\overline{\mathbf{x}}_{1, unbiased} =\\
&\overline{\mathbf{x}}_{2, unbiased} =\\
&\widehat{\boldsymbol{\Sigma}}_{1, unbiased} =\\
&\widehat{\boldsymbol{\Sigma}}_{2, unbiased} =\\
&\widehat{\boldsymbol{\Sigma}}_{1, biased} =\\
&\widehat{\boldsymbol{\Sigma}}_{2, biased} =\\
\end{align*}
The code is attached.
\end{answer}
\end{question}


%----------------------------------------------

\begin{question}{Class Density}{5}
Using the unbiased estimates from the previous question, fit a Gaussian distribution to the data of each class. Generate a single plot showing the data points and the probability densities of each class.
(Hint: use the contour function for plotting the Gaussians.) 

\begin{answer}\end{answer}

\end{question}

%----------------------------------------------

\begin{question}{Posterior}{8}
In a single graph, plot the posterior distribution of each class $p(C_i|x)$ and show the decision boundary. 

\begin{answer}\end{answer}

\end{question}

%----------------------------------------------

\begin{question}[bonus]{Bayesian Estimation}{15}
State the generic case of Bayesian linear regression with data $<\vec X, \vec Y>$ and parameters $\vec \theta$. What do we assume about the data, the model and the parameters?\\
Formulate the posterior distribution for your model parameters given the data, i.e., $p(\vec \theta | \vec X, \vec Y)$, and derive its mean and covariance, assuming that the model of the output variable is a Gaussian distribution with a fixed variance.\\
What do we do when we want to predict a new point?\\
Which are the advantages of being Bayesian? 

\begin{answer}\end{answer}

\end{question}

\end{questions}
